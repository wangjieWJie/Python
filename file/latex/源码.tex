\documentclass{article}
\usepackage{CJKutf8}

\begin{document}

\begin{CJK*}{UTF8}{gbsn}


A-1、

证明:(1) LS耦合

\( S = 0.1; L = 5,4,3,2,1 \)

\( S = 0时; J = L \)

5个L值分别得出5个J值,即5个单重态。

\( S = 1时; J = L+1, L, L-1 \)

代入一个L值便有一个三重态。5个L值共有5乘3等于15个原子态:
\[
^3P_{0,1,2}, ^3D_{1,2,3}, ^3F_{2,3,4}, ^3G_{3,4,5}, ^3H_{4,5,6}
\]

因此,LS耦合时共有20个可能的状态。

(2) jj耦合:

\[
j = l + s或j = l - s; j_1 = \frac{5}{2}或\frac{3}{2}, j_2 = \frac{7}{2}或\frac{5}{2}
\]

\[
J = j_1 + j_2, j_1 + j_2 - 1, \ldots, |j_1 - j_2|
\]

将每个\(j_1, j_2\) 合成J得:

\[
j_1 = \frac{5}{2} 和 j_2 = \frac{7}{2}, 合成J = 6, 5, 4, 3, 2, 1
\]

\[
j_1 = \frac{3}{2} 和 j_2 = \frac{7}{2}, 合成J = 5, 4, 3, 2
\]

\[
j_1 = \frac{5}{2} 和 j_2 = \frac{5}{2}, 合成J = 5, 4, 3, 2, 1, 0
\]

\[
j_1 = \frac{3}{2} 和 j_2 = \frac{5}{2}, 合成J = 4, 3, 2, 1
\]

共20个状态:\((\frac{5}{2} \frac{7}{2})_{6, 5, 4, 3, 2, 1}, (\frac{3}{2} \frac{7}{2})_{5, 4, 3, 2}, (\frac{5}{2} \frac{5}{2})_{5, 4, 3, 2, 1, 0}, (\frac{3}{2} \frac{5}{2})_{4, 3, 2, 1}\)

所以,对于相同的组态无论是LS耦合还是jj耦合,都会给出同样数目的可能状态。

A-2、

解:
子弹的动量 $p = mv = 2
\mathrm{kg}\cdot\mathrm{m}\cdot\mathrm{s}^{-1}$


动量的不确定范围
$$\Delta p = 0.01\% \times p = 2 \times 10^{-4}\, \mathrm{kg}\cdot\mathrm{m}\cdot\mathrm{s}^{-1}$$

位置的不确定范围

$\Delta x \geq  \frac{h}{4\pi \cdot \Delta p} = \frac{1.05 \times 10^{-34} J \cdot S }{2\times 2 \times \10^{-4} kg \cdot m \cdot s^{-1}}$
= 2.625 \times 10^{-31} m  

A-3、

解:电子的全部能量转换为光子的能量时,X光子的波长最短。光子的最大能量是:
\[
\varepsilon_{\max} = Ve = 10^5 \text{ eV}
\]

而
\[
\varepsilon_{\max} = h \frac{c}{\lambda_{\min}}
\]

所以
\[
\lambda_{\min} = h \frac{c}{\varepsilon_{\max}} = \frac{6.63 \times 10^{-34} \times 3 \times 10^8}{10^5 \times 1.60 \times 10^{-19}} = 0.124 \text{nm}
\]

A-4、

解: 由式 (38-3) 知α衰变能为
\[ Q_{\alpha} = \left[ M_{X} - \left( M_{Y} + M_{\mathrm{He}} \right) \right]c^{2} \]
式中 \( M_x, M_Y \) 和 \( M_\mathrm{He} \) 分别为母核、子核和氦的原子质量

对衰变

\[ ^{226}Ra \rightarrow ^{222}Ra + \alpha \]

已知 \( M_{x} = M_{\mathrm{Ra}} = 226.0254 \, \text{u}, M_{Y} = M_{\mathrm{Rn}} = 222.0176 \, \text{u}, M_{\mathrm{He}} = 4.002603 \, \text{u}, \text{代人计} \)

算得 α 衰变能为
\[ 
\begin{aligned}
& Q_{\alpha} = \left[ M_{\mathrm{Ra}} - \left( M_{\mathrm{Ra}} + M_{\mathrm{Ha}} \right) \right]c^{2} \\
& = 226.0254
\end{aligned}
\]

根据动量守恒得到子核的反冲动能
\[ E_{r} = \frac{m_{\alpha}}{m_{\gamma}}E_{\alpha} \]

根据能量守恒定律,α 衰变能还等于α粒子的动能和子核的反冲动能之和。根据式 (38-5) 知α衰变能为

\[ Q_{\alpha} = E_{\alpha} + E_{r} = \left(1 + \frac{m_{\alpha}}{m_{y}}\right)E_{\alpha} \approx \frac{A}{A-4}E_{\alpha} \]
式中 \( A \) 为 \( X \) 的原子量, \( A = 226 \)。则发射的 \( \alpha \) 粒子的能量为
\[ E_{\alpha} = Q_{\alpha} \frac{A-4}{A} = 4.84 \, \text{MeV} \times \frac{226-4}{226} = 4.75 \, \text{MeV} \]

 A-5、
 
解: \( Li^+ \) 从 第 一 激 发 态 向 基 态 跃 迁 时 发 出 光 子 的 能 量 为


$\mathrm{He}^{+}$的电离能为
$$h\nu_{\mathrm{Li}}=9hcR_{\mathrm{Li}}\Big|\:\frac{1}{1^{2}}-\:\frac{1}{2^{2}}\Big|=\:\frac{27}{4}RcR_{\mathrm{Li}}$$
$$h\nu_{\mathrm{He}}=4hcR_{\mathrm{He}}\Big[\:\frac{1}{1^{2}}-\frac{1}{\infty}\Big]=4hcR_{\mathrm{He}}$$
二者相比得

$$\frac{h\nu_{i,s}}{h\nu_{\mathrm{He}}}=\frac{27R_{i,s}}{16}\frac{27}{R_{\mathrm{He}}}=\frac{27}{16}\cdot\frac{1+m/M_{\mathrm{He}}}{1+m/M_{\mathrm{Li}}}$$
由于 $M_{\mathrm{te}}<M_{\mathrm{Li}}$,所以有 1+$m/M_{\mathrm{He}}>1+m/M_{\mathrm{Ui}}$,从而有
$$h\nu_{\mathrm{Li}}>h\nu_{\mathrm{He}}$$
由此知,\( Li^+ \) 放出的光子可电离基态的 \( He^+ \)离子.

A-6、

解:$I_1= 0, I_2= 1, s_1= s_2= 1/ 2; S= 0, 1; L= 1$

对于$S=0,J=L=1$ ,单态 1Pl

对于$S=1,J=2,1,0$ ,三重态 3P$_2.1.0$

根据选择定则,可能出现 5 条谱线,它们分别由下列跃迁产生:2$\mathrm{P}_1\to1^1\mathrm{S}_0;2^1\mathrm{P}_1\to2^1\mathrm{S}_0$
$$2^3\mathrm{P}_0\to2^3\mathrm{S}_1;\:2^3\mathrm{P}_1\to2^3\mathrm{S}_1;\:2^3\mathrm{P}_2\to2^3\mathrm{S}_1$$

A-7、
解:
(1) 
已知原子态为 $^3D$,电子组态为 2p3d,则
\begin{align*}
L &= 2, S &= 1, I_1 &= 1, I_2 &= 2.
\end{align*}

因此

\begin{align*}
&p_{\mu_{1}}=\sqrt{I_{1}(I_{1}+1)}\frac{h}{2\pi}=\sqrt{2}h, \\
&p_{\mu_{2}}=\sqrt{I_{2}(I_{2}+1)}h=\sqrt{6}h, \\
&P_{z}=\sqrt{L(L+1)}h=\sqrt{6}h, \\
&P_{z}^{2}=p_{\mu_{1}}^{2}+p_{\mu_{2}}^{2}+2p_{\mu_{1}}p_{\mu_{2}}\cos\theta_{L}.
\end{align*}

因此

\begin{align*}
\cos\theta_{z}&=\frac{(P_{z}^{2}-p_{\mu_{1}}^{2}-p_{\mu_{2}}^{2})}{2p_{\mu}p_{\mu_{2}}}=-\frac{1}{2\sqrt{3}}, \\
\theta_{z}&=106^{\circ}46^{\circ}.
\end{align*}

因为

$s_{1}=s_{2}=\frac{1}{2}$,所以
\begin{align*}
&\rho_{1}=\rho_{2}=\sqrt{s(s+1)}h=\frac{\sqrt{3}}{2}h, \\
&P_{s}=\sqrt{S(S+1)}h=\sqrt{2}h, \\
&P_{s}^{2}={\rho_{\mu_{1}}}^{2}+{\rho_{\mu_{2}}}^{2}+2{\rho_{\mu_{1}}\rho_{\nu_{2}}}\cos\theta_{s}.
\end{align*}

因此

\begin{align*}
\cos\theta_{s}&=\frac{(P_{s}^{2}-{\rho_{\mu_{1}}}^{2}-\rho_{\mu_{2}}^{2})}{2{\rho_{\mu_{1}}\rho_{\nu_{2}}}}=\frac{1}{3}, \\
\theta_{s}&=70^{\circ}32^{\circ}.
\end{align*}

(2)
\begin{align*}
&\because s_{1}=s_{2}=\frac{1}{2}, \\
&\therefore p_{1}=p_{2}=\sqrt{s(s+1)}h=\frac{\sqrt{3}}{2}h, \\
&P_{s}=\sqrt{S(S+1)}h=\sqrt{2}h, \\
&P_{S}^{2}={p_{s1}}^{2}+{p_{s2}}^{2}+2p_{s1}p_{s2}\cos\theta_{s}, \\
&\therefore\cos\theta_{s}=(P_{S}^{2}-{p_{s1}}^{2}-{p_{s2}}^{2})/2p_{s1}p_{s2}=\frac{1}{3}, \\
&\theta_{S}=70^{\circ}32'.
\end{align*}

A-8、
解: 由公式(1-3),散射角大于 90°的α粒子数为
$$\mathrm{d}n^{\prime}=\int\:\mathrm{d}n=nNt\int_{\pi/2}^{\pi}\mathrm{d}\sigma $$
### 所以,占总数的相对数为
$$\frac{\mathrm{d}n^{\prime}}{n}=Nt\int_{\pi/2}^{\pi}\mathrm{d}\sigma=Nt\pi\Big|\frac{1}{4\pi\varepsilon_{0}}\Big|^{2}(\frac{Ze^{2}}{T})^{2}\Big|_{\pi/2}^{\pi}\frac{\cos(\theta/2)}{\sin^{3}(\theta/2)}\mathrm{d}\theta $$
其中,单位体积中金的原子数为

\[ N = \frac{\rho N_{0}}{A} = \frac{1.93 \times 10^{4} \, \text{kg} / \text{m}^{3} \times 6.02 \times 10^{29} \times 10^{3} \, \text{kmol^{-1}}}{197 \, \text{kg} / \text{kmol}} \]

\[ = 5.9 \times 10^{28} \, \text{m}^{-3} \]

\[
\left\langle\frac{1}{4\pi\epsilon_{0}}\right\rangle^{2}\left(\frac{Ze^{2}}{T}\right)^{t} = \left[\frac{8.98\times10^{9}\text{ N}\cdot\text{m}^{2}/\text{C}^{2}\times79\times(1.602\times10^{-10}\text{C})^{z}}{8.8\times10^{6}\text{ eV}\times1.602\times10^{-19}\text{J/eV}}\right]^{z}
=1.69\times10^{-28}\text{米}^{2}.
\]

\[
I = \int_{\pi/2}^{\pi}\frac{\cos(\theta/2)}{\sin^{3}(\theta/2)}\mathrm{d}\theta = 2\int_{\pi/2}^{\pi}\frac{\mathrm{d}\sin(\theta/2)}{\sin^{3}(\theta/2)} = 1
\]
将以上数值代入,即得
$$\begin{aligned}\frac{\mathrm{d}n^{\prime}}{n}&=5.9\times10^{28}m^{-3}\times2\times10^{-7}m\times1.69\times10^{-28}m^{2}\\&=6.25\times10^{-4}\%\end{aligned}$$

A-9、

解: 散射光子能量为
$$h\nu^{\prime}=\frac{h\nu}{1+\frac{h\nu\left(1-\cos\theta\right)}{m_{0}c^{2}}}=h\nu\frac{m_{0}c^{2}}{m_{0}c^{2}+h\nu\left(1-\cos\theta\right)}<h\nu\frac{m_{0}c^{2}}{h\nu\left(1-\cos\theta\right)}$$
若 $\theta>60^{\circ}$,则 $\cos\theta<\frac12$,代人上式可得
$$h\nu^{\prime}<2m_{0}c^{2}$$
所以,散射光子总不能再产生正负电子偶

A-10、

解 (1) 钾原子基态电子组态 4s;原子态为 4^{2}S_{1/2}.

第一激发态为4p,对应的原子组态为 \( 4^2\mathrm{P}_{3/2},4^2\mathrm{P}_{1/2}; \) 第一激发态向基态跃迁为 \( 4^{2}P_{3/2}\to4^{2}S_{1/2},4^{2}P_{1/2}\to4^{2}S_{1/2}. \)

对原子态 \( 4^2\mathbb{P}_{3/2} \),其对应的角动量量子数为
\[
\begin{aligned}
s &= \frac{1}{2}; & l &= 1; & j &= \frac{3}{2}, & m_{j} &= \pm\frac{3}{2},\pm\frac{1}{2}
\end{aligned}
\]

\[ f(x) = \frac{1}{2} \]

朗德 \( g \) 因子为

\[
\begin{aligned}
g_{j} &= \frac{3}{2} + \frac{s(s+1)-l(l+1)}{2j(j+1)} \\
&= \frac{3}{2} + \frac{\frac{1}{2}\left(\frac{1}{2}+1\right)-1(1+1)}{2\times\frac{3}{2}\left(\frac{3}{2}+1\right)} \\
&= \frac{4}{3}
\end{aligned}
\]

则有、
\[ m_{j}g_{j} = \pm\frac{6}{3},\pm\frac{2}{3} \]

对原子态 \( 4^2\mathbb{P}_{1/2} \) 其对应的角动量量子数为
\[
\begin{aligned}
s &= \frac{1}{2}; & l &= 1; & j &= \frac{1}{2}, & m_{j} &= \pm\frac{1}{2}
\end{aligned}
\]

朗德 \( g \) 因子为
\[
\begin{aligned}
g_{j} &= \frac{3}{2} + \frac{s(s+1)-l(l+1)}{2j(j+1)} \\
&= \frac{3}{2} + \frac{\frac{1}{2}\left(\frac{1}{2}+1\right)-1(1+1)}{2\times\frac{1}{2}\left(\frac{1}{2}+1\right)} \\
&= \frac{2}{3}
\end{aligned}
\]

则有
\[ m_{j}g_{j} = \pm\frac13 \]

对原子态 \( 4^2S_{1/2} \) 其对应的角动量量子数为
\[
\begin{aligned}
s &= \frac{1}{2}; & l &= 0; & j &= \frac{1}{2}; & m_{j} &= \pm\frac{1}{2}
\end{aligned}
\]

朗德 \( g \) 因子为 \( g=2 \),则有 \( m_jg_j=\pm1. \)

在磁场中,引起的附加能量为
\[ E=-\mu_{j}\cdot B=-\mu_{ja}B=g_{j}m_{j}\mu_{B}B \]

所以 \( 4^2\mathrm{P}_{3/2} \) 能级分裂的能级间距为
\[ \Delta E_{2}^{\prime} = \frac{4}{3}\mu_{B}B \]

\( 4^2P_{1/2} \) 能级分裂的能级间距为
\[ \Delta E_{1}^{\prime} = \frac{2}{3}\mu_{B}B \]

\( 4^{2}S_{1/2} \) 能级分裂的能级间距为
\[ \Delta E_{0}^{\prime} = 2\mu_{B}B \]

(2) 由能级图可以看出分裂后的最高与最低能级差 \( \Delta E_2 \) 与原能级差 \( \Delta E_1 \)

的关系为
\[ 
\begin{aligned}
\Delta E_{2} &= \Delta E_{1} + 1.5\Delta E_{2}^{\prime} + 0.5\Delta E_{1}^{\prime} \\
&= \Delta E_{1} + 1.5\times\frac{4}{3}\mu_{B}B + 0.5\times\frac{2}{3}\mu_{B}B \\
&= \Delta E_{1} + \frac{7}{3}\mu_{B}B
\end{aligned}
\]

已知精细结构谱线分别为 \( \lambda_{1} = 766.4 \) nm 和 \( \lambda_{2} = 769.9 \) nm,则能量差 \( \Delta E_{_1} \) 又可

以表示为

\[ \Delta E_{1} = \frac{hc}{\lambda_{1}} - \frac{hc}{\lambda_{2}} = \frac{hc(\lambda_{2}-\lambda_{1})}{\lambda_{1}\lambda_{2}} \]

已知 \( \Delta E_{2} = 1.5\Delta E_{1} \),则有
\[ \frac{7}{3}\mu_{_B}B = 0.5\Delta E_{_1} \]

由此得所加磁场 \( B \) 为
\[ B = \frac{0.5\Delta E_{1}}{\frac{7}{3}\mu_{B}} = \frac{0.5hc\left(\lambda_{2}-\lambda_{1}\right)}{\lambda_{1}\lambda_{2}\frac{7}{3}\mu_{B}} \]
\[ = \frac{0.5\times1.24\mathrm{~nm~}\cdot\mathrm{keV}\times\left(769.9\mathrm{~nm-766.4~nm}\right)}{769.9\mathrm{~nm}\times766.4\mathrm{~nm}\times\frac{7}{3}\times0.5788\times10^{-4}\mathrm{eV}\cdot T^{-1}} \]

\[  = 27.2T \]


\end{CJK*}


\end{document}
